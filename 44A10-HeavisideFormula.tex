\documentclass[12pt]{article}
\usepackage{pmmeta}
\pmcanonicalname{HeavisideFormula}
\pmcreated{2014-03-19 9:14:46}
\pmmodified{2014-03-19 9:14:46}
\pmowner{pahio}{2872}
\pmmodifier{pahio}{2872}
\pmtitle{Heaviside formula}
\pmrecord{11}{40909}
\pmprivacy{1}
\pmauthor{pahio}{2872}
\pmtype{Topic}
\pmcomment{trigger rebuild}
\pmclassification{msc}{44A10}
\pmsynonym{Heaviside expansion formula}{HeavisideFormula}
\pmsynonym{inverse Laplace transform of rational function}{HeavisideFormula}
\pmrelated{HyperbolicFunctions}
\pmrelated{ComplexSineAndCosine}

% this is the default PlanetMath preamble.  as your knowledge
% of TeX increases, you will probably want to edit this, but
% it should be fine as is for beginners.

% almost certainly you want these
\usepackage{amssymb}
\usepackage{amsmath}
\usepackage{amsfonts}

% used for TeXing text within eps files
%\usepackage{psfrag}
% need this for including graphics (\includegraphics)
%\usepackage{graphicx}
% for neatly defining theorems and propositions
 \usepackage{amsthm}
% making logically defined graphics
%%%\usepackage{xypic}

% there are many more packages, add them here as you need them

% define commands here

\theoremstyle{definition}
\newtheorem*{thmplain}{Theorem}

\begin{document}
Let $P(s)$ and $Q(s)$ be polynomials with the degree of the former less than the degree of the latter.
\begin{itemize}
\item If all complex \PMlinkname{zeroes}{Zero} $a_1,\,a_2,\,\ldots,\,a_n$ of $Q(s)$ are simple, then
\begin{align}
\mathcal{L}^{-1}\left\{\frac{P(s)}{Q(s)}\right\} = \sum_{j=1}^n\frac{P(a_j)}{Q'(a_j)}e^{a_jt}.
\end{align}
\item If the different zeroes $a_1,a_2,\,\ldots,a_n$ of $Q(s)$ 
have the multiplicities $m_1,m_2,\,\ldots,m_n$, respectively, 
we denote\, $F_j(s) := (s\!-\!a_j)^{m_j}P(s)/Q(s)$;\, then
\begin{align}
\mathcal{L}^{-1}\left\{\frac{P(s)}{Q(s)}\right\} = 
\sum_{j=1}^ne^{a_jt}\sum_{k=0}^{m_j-1}\frac{F_j^{(k)}(a_j)t^{m_j\!-\!1\!-\!k}}{k!(m_j\!-\!1\!-\!k)!}.
\end{align}
\end{itemize}
 
A special case of the {\em Heaviside formula} (1) is
\begin{align}
\mathcal{L}^{-1}\left\{\frac{Q'(s)}{Q(s)}\right\} 
\;=\; \sum_{j=1}^ne^{a_jt}.
\end{align}\\

\textbf{Example.}\, Since the zeros of the binomial $s^4\!+\!4a^4$ are\, $s = (\pm1\!\pm\!i)a$,\, we 
can calculate by (3) as follows:
$$\mathcal{L}^{-1}\left\{\frac{s^3}{s^4\!+\!4a^4}\right\} 
= \frac{1}{4}\mathcal{L}^{-1}\left\{\frac{4s^3}{s^4\!+\!4a^4}\right\} = \frac{1}{4}\sum_\pm e^{(\pm 1\pm i)at} 
= \frac{e^{at}+e^{-at}}{2}\cdot\frac{e^{iat}+e^{-iat}}{2} 
= \cosh{at}\,\cos{at}$$\\


{\it Proof of (1).}\, Without hurting the generality, we can 
suppose that $Q(s)$ is monic.\, Therefore
$$Q(s) = (s\!-\!a_1)(s\!-\!a_2)\cdots(s\!-\!s_n).$$
For\, $j = 1,2,\;\ldots,\,n$,\, denoting
$$Q(s) := (s\!-\!a_j)Q_j(s),$$
one has\, $Q_j(a_j) \neq 0$.\, We have a partial fraction 
expansion of the form
\begin{align}
\frac{P(s)}{Q(s)} = \frac{C_1}{s\!-\!a_1}+\frac{C_2}{s\!-\!a_2}+\ldots+\frac{C_n}{s\!-\!a_n}
\end{align}
with constants $C_j$.\, According to the linearity and the 
formula 1 of the \PMlinkname{parent entry}{LaplaceTransform}, 
one gets
\begin{align}
\mathcal{L}^{-1}\left\{\frac{P(s)}{Q(s)}\right\} 
= \sum_{j=1}^nC_je^{a_jt}.
\end{align}
For determining the constants $C_j$, multiply (3) by 
$s\!-\!a_j$.\, It yields
$$\frac{P(s)}{Q_j(s)} 
= C_j+(s\!-\!a_j)\sum_{\nu \neq j}\frac{C_\nu}{s\!-\!a_\nu}.$$
Setting to this identity \,$s := a_j$\, gives the value
\begin{align}
C_j = \frac{P(a_j)}{Q_j(a_j)}.
\end{align}
But since\, $Q'(s) = \frac{d}{ds}((s\!-\!a_j)Q_j(s)) 
= Q_j(s)\!+\!(s\!-\!a_j)Q_j'(s)$,\, 
we see that\, $Q'(a_j) = Q_j(a_j)$;\, thus the equation (5) may 
be written
\begin{align}
C_j = \frac{P(a_j)}{Q'(a_j)}.
\end{align}
The values (6) in (4) produce the formula (1).

\begin{thebibliography}{9}
\bibitem{K.V.}{\sc K. V\"ais\"al\"a:} {\em Laplace-muunnos}.\, Handout Nr. 163.\quad Teknillisen korkeakoulun ylioppilaskunta, Otaniemi, Finland (1968).
\end{thebibliography}
%%%%%
%%%%%
\end{document}
