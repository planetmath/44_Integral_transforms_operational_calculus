\documentclass[12pt]{article}
\usepackage{pmmeta}
\pmcanonicalname{LaplaceTransformOfLogarithm}
\pmcreated{2013-03-22 18:26:01}
\pmmodified{2013-03-22 18:26:01}
\pmowner{pahio}{2872}
\pmmodifier{pahio}{2872}
\pmtitle{Laplace transform of logarithm}
\pmrecord{7}{41091}
\pmprivacy{1}
\pmauthor{pahio}{2872}
\pmtype{Theorem}
\pmcomment{trigger rebuild}
\pmclassification{msc}{44A10}
\pmsynonym{Laplace transform of logarithm function}{LaplaceTransformOfLogarithm}
\pmrelated{PowerFunction}

\endmetadata

% this is the default PlanetMath preamble.  as your knowledge
% of TeX increases, you will probably want to edit this, but
% it should be fine as is for beginners.

% almost certainly you want these
\usepackage{amssymb}
\usepackage{amsmath}
\usepackage{amsfonts}

% used for TeXing text within eps files
%\usepackage{psfrag}
% need this for including graphics (\includegraphics)
%\usepackage{graphicx}
% for neatly defining theorems and propositions
 \usepackage{amsthm}
% making logically defined graphics
%%%\usepackage{xypic}

% there are many more packages, add them here as you need them

% define commands here

\theoremstyle{definition}
\newtheorem*{thmplain}{Theorem}

\begin{document}
\textbf{Theorem.}\; The Laplace transform of the natural logarithm function is
$$\mathcal{L}\{\ln{t}\} \;=\; \frac{\Gamma\,'(1)-\ln{s}}{s}$$
where $\Gamma$ is Euler's gamma function.\\

{\em Proof.}\; We use the \PMlinkname{Laplace transform of the power function}{LaplaceTransformOfPowerFunction}
$$\int_0^\infty e^{-st}t^a\,dt \;=\; \frac{\Gamma(a\!+\!1)}{s^{a+1}}$$
by differentiating it with respect to the parametre $a$:
$$\int_0^\infty e^{-st}t^a\ln{t}\;dt \;=\; \frac{\Gamma'(a\!+\!1)s^{a+1}-\Gamma(a\!+\!1)s^{a+1}\ln{s}}{(s^{a+1})^2}
\;=\; \frac{\Gamma'(a\!+\!1)-\Gamma(a\!+\!1)\ln{s}}{s^{a+1}}$$
Setting here\, $a = 0$,\, we obtain
$$\mathcal{L}\{\ln{t}\} \;=\; \int_0^\infty e^{-st}\ln{t}\;dt \;=\; \frac{\Gamma\,'(1)-1\cdot\ln{s}}{s},$$
Q.E.D.\\

\textbf{Note.}\; The number $\Gamma'(1)$ is equal the \PMlinkescapetext{negative} of the \PMlinkname{Euler--Mascheroni constant}{EulersConstant}, as is seen in the entry digamma and polygamma functions.



%%%%%
%%%%%
\end{document}
