\documentclass[12pt]{article}
\usepackage{pmmeta}
\pmcanonicalname{KacsTheorem}
\pmcreated{2014-03-19 22:18:04}
\pmmodified{2014-03-19 22:18:04}
\pmowner{Filipe}{28191}
\pmmodifier{Filipe}{28191}
\pmtitle{Kac's theorem}
\pmrecord{4}{88071}
\pmprivacy{1}
\pmauthor{Filipe}{28191}
\pmtype{Theorem}
\pmrelated{Poincaré Recurrence theorem}

\endmetadata

% this is the default PlanetMath preamble.  as your knowledge
% of TeX increases, you will probably want to edit this, but
% it should be fine as is for beginners.

% almost certainly you want these
\usepackage{amssymb}
\usepackage{amsmath}
\usepackage{amsfonts}

% need this for including graphics (\includegraphics)
\usepackage{graphicx}
% for neatly defining theorems and propositions
\usepackage{amsthm}

% making logically defined graphics
%\usepackage{xypic}
% used for TeXing text within eps files
%\usepackage{psfrag}

% there are many more packages, add them here as you need them

% define commands here

\begin{document}
Let $f:M\rightarrow M$ be a transformation and $\mu$ a finite invariant measure for $f$. Let $E$ be a subset of $M$ with positive measure. We define the first return map for $E$:
$$\rho_E(x)=\min \{ n \geq 1: f^n(x)\in E \}$$
If the set on the right is empty, then we define $\rho_E(x)=\infty$. The Poincaré recurrence theorem asserts that $\rho_E$ is finite for almost every $x \in R$.
We define the following sets:
$$E_0 = \{ x \in E: f^n(x) \notin E, n\geq 1 \}$$
$$E_0^*= \{ x \in M: f^n(x) \notin E, n \geq 0 \}$$
By Poincaré recurrence theorem, $\mu(E_0)=0$.
Kac's theorem asserts that the function $\rho_E$ is integrable and
$$\int_E \rho_E d\mu= \mu(M)-\mu(E_0^*)$$
When the system is ergodic, then $\mu(E_0^*)=0$, and Kac's theorem implies:
$$\frac{1}{\mu(E)}\int_E \rho_E d\mu = \frac{\mu(M)}{\mu(E)}$$
This equality can be interpreted as: the mean return time to $E$ s inversely proportional to the measure of $E$.
\end{document}
