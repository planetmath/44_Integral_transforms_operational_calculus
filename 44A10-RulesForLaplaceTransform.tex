\documentclass[12pt]{article}
\usepackage{pmmeta}
\pmcanonicalname{RulesForLaplaceTransform}
\pmcreated{2013-03-22 18:31:08}
\pmmodified{2013-03-22 18:31:08}
\pmowner{pahio}{2872}
\pmmodifier{pahio}{2872}
\pmtitle{rules for Laplace transform}
\pmrecord{7}{41206}
\pmprivacy{1}
\pmauthor{pahio}{2872}
\pmtype{Derivation}
\pmcomment{trigger rebuild}
\pmclassification{msc}{44A10}
\pmrelated{TableOfLaplaceTransforms}

% this is the default PlanetMath preamble.  as your knowledge
% of TeX increases, you will probably want to edit this, but
% it should be fine as is for beginners.

% almost certainly you want these
\usepackage{amssymb}
\usepackage{amsmath}
\usepackage{amsfonts}

% used for TeXing text within eps files
%\usepackage{psfrag}
% need this for including graphics (\includegraphics)
%\usepackage{graphicx}
% for neatly defining theorems and propositions
 \usepackage{amsthm}
% making logically defined graphics
%%%\usepackage{xypic}

% there are many more packages, add them here as you need them

% define commands here

\theoremstyle{definition}
\newtheorem*{thmplain}{Theorem}

\begin{document}
If\, $\mathcal{L}\{f(t)\} = F(s)$,\, then
\begin{itemize}
\item $\mathcal{L}\{e^{at}f(t)\} \,=\, F(s\!-\!a)$ \quad for\; $s > a$,
\item $\mathcal{L}\{f(\frac{t}{a})\} \;=\; a\,F(as)$ \qquad for\; $a > 0$.\\
\end{itemize}

For deriving these rules, we start from the definition of Laplace transform.\, In the first case, we shall use the notation \,$s\!-\!a = r$:
$$\mathcal{L}\{e^{at}f(t)\} = \int_0^\infty\!e^{-st}e^{at}f(t)\,dt = 
\int_0^\infty\!e^{-(s-a)t}f(t)\,dt = \int_0^\infty\!e^{-rt}f(t)\,dt = F(r) = F(s\!-\!a).$$ 
In the second case, we make the change of variable \,$\frac{t}{a} = u$\, and later use the notation\, $sa = r$:
$$\mathcal{L}\{f(\frac{t}{a})\} = \int_0^\infty\!e^{-st}f(\frac{t}{a})\,dt =
a\!\int_0^\infty\!e^{-sau}f(u)\,du = a\!\int_0^\infty\!e^{-ru}f(u)\,du = aF(r) = a\,F(as).$$
%%%%%
%%%%%
\end{document}
