\documentclass[12pt]{article}
\usepackage{pmmeta}
\pmcanonicalname{LaplaceTransform}
\pmcreated{2014-03-10 10:50:28}
\pmmodified{2014-03-10 10:50:28}
\pmowner{rspuzio}{6075}
\pmmodifier{pahio}{2872}
\pmtitle{Laplace transform}
\pmrecord{26}{34343}
\pmprivacy{1}
\pmauthor{rspuzio}{2872}
\pmtype{Definition}
\pmcomment{trigger rebuild}
\pmclassification{msc}{44A10}
\pmrelated{DiscreteFourierTransform}
\pmrelated{UsingLaplaceTransformToInitialValueProblems}
\pmrelated{UsingLaplaceTransformToSolveHeatEquation}

\endmetadata

% this is the default PlanetMath preamble.  as your knowledge
% of TeX increases, you will probably want to edit this, but
% it should be fine as is for beginners.

% almost certainly you want these
\usepackage{amssymb}
\usepackage{amsmath}
\usepackage{amsfonts}

% used for TeXing text within eps files
%\usepackage{psfrag}
% need this for including graphics (\includegraphics)
%\usepackage{graphicx}
% for neatly defining theorems and propositions
%\usepackage{amsthm}
% making logically defined graphics
%%%\usepackage{xypic}

% there are many more packages, add them here as you need them

% define commands here
\begin{document}
Let $f(t)$ be a function defined on the interval \,$[0,\,\infty)$.  The
\emph{Laplace transform} of $f(t)$ is the function $F(s)$ defined by
\[
F(s)\,=\,\int_{0}^{\infty}e^{-st} f(t)\,dt,
\]
provided that the integral converges. \footnote{Depending on the definition of
integral one is using, one may prefer to define the Laplace transform as
$\lim_{x \to 0+} \int_{x}^{\infty}e^{-st} f(t)\,dt$}  It suffices that $f$ be defined when $t>0$ and $s$ can be complex.  We will 
usually denote the Laplace transform of $f$ by $\mathcal{L}\{f\}$.  Some
of the most common Laplace transforms are:

\begin{enumerate}
\item $\displaystyle\mathcal{L}\{e^{at}\}\,=\,\frac{1}{s-a},\;\; s>a$

\item $\displaystyle\mathcal{L}\{\cos(bt)\}\,=\,\frac{s}{s^{2}+b^{2}},\;\; s>0$

\item $\displaystyle\mathcal{L}\{\sin(bt)\}\,=\,\frac{b}{s^{2}+b^{2}},\;\; s>0$

\item $\displaystyle\mathcal{L}\{t^{n}\}\,=\,\frac{\Gamma(n+1)}{s^{n+1}},\;\; s>0,\; n>-1.$

\item $\displaystyle\mathcal{L}\{f'\}\,=\, s\mathcal{L}\{f\}-\lim_{x \to 0+}f(x)$
\end{enumerate}

For more particular Laplace transforms, see the table of Laplace 
transforms.\\

Notice the Laplace transform is a linear transformation.  It is worth noting that, if 
 $$\int_{0}^{\infty}e^{-st}|f(t)|\,dt < \infty$$
for some\, $s \in \mathbb{R}$, then $\mathcal{L}\{f\}$ is an analytic function in the complex half-plane 
$\{z \mid\; \Re z > s\}$.

Much like the Fourier transform, the Laplace transform has a convolution.  However, the form of the convolution used is different.
$$\mathcal{L}\{f*g\} = \mathcal{L}\{f\} \mathcal{L}\{g\}$$
where
$$(f*g) (t) = \int_0^t f(t-s) g(s) \, ds$$
and
$$\mathcal{L}\{fg\}(s) = \int_{c - i \infty}^{c + i \infty} \mathcal{L}\{f\}(z) \mathcal{L}\{g\}(s-z) \, dz$$

The most popular usage of the Laplace transform is to solve 
initial value problems by taking the Laplace transform of both 
sides of an ordinary differential equation; see the entry 
``\PMlinkname{image equation}{ImageEquation}''.
%%%%%
%%%%%
\end{document}
