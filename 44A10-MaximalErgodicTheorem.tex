\documentclass[12pt]{article}
\usepackage{pmmeta}
\pmcanonicalname{MaximalErgodicTheorem}
\pmcreated{2014-03-19 22:15:48}
\pmmodified{2014-03-19 22:15:48}
\pmowner{Filipe}{28191}
\pmmodifier{Filipe}{28191}
\pmtitle{Maximal ergodic theorem}
\pmrecord{3}{88066}
\pmprivacy{1}
\pmauthor{Filipe}{28191}
\pmtype{Theorem}
\pmrelated{birkhoff ergodic theorem}

\endmetadata

% this is the default PlanetMath preamble.  as your knowledge
% of TeX increases, you will probably want to edit this, but
% it should be fine as is for beginners.

% almost certainly you want these
\usepackage{amssymb}
\usepackage{amsmath}
\usepackage{amsfonts}

% need this for including graphics (\includegraphics)
\usepackage{graphicx}
% for neatly defining theorems and propositions
\usepackage{amsthm}

% making logically defined graphics
%\usepackage{xypic}
% used for TeXing text within eps files
%\usepackage{psfrag}

% there are many more packages, add them here as you need them

% define commands here

\begin{document}
Let $(X,\mathcal{B},\mu)$ be a probability space and $T:X\rightarrow X$ a measure preserving transformation. Let $f$ be a $L^1(\mu)$ function.
Define the averages 
$$f^*(x)=\sup_{N\geq 1} \frac{1}{N} \sum_{i=0}^{N-1} f(T^i(x))$$
Then, for any $\lambda \in \textbf{R}$, we have:
$$\int_{f^*>\lambda} fd\mu \geq \lambda \mu ( \{f^* > \lambda \} )$$
This theorem may be used in the proof of the ergodic theorem (also known as Birkhoff ergodic theorem, or pointwise or strong ergodic theorem)
\end{document}
