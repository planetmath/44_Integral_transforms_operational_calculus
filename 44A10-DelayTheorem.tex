\documentclass[12pt]{article}
\usepackage{pmmeta}
\pmcanonicalname{DelayTheorem}
\pmcreated{2013-03-22 18:02:46}
\pmmodified{2013-03-22 18:02:46}
\pmowner{pahio}{2872}
\pmmodifier{pahio}{2872}
\pmtitle{delay theorem}
\pmrecord{10}{40570}
\pmprivacy{1}
\pmauthor{pahio}{2872}
\pmtype{Theorem}
\pmcomment{trigger rebuild}
\pmclassification{msc}{44A10}
\pmsynonym{delay theorem of Laplace transform}{DelayTheorem}
%\pmkeywords{Laplace transform}
\pmrelated{HeavisideStepFunction}
\pmrelated{TelegraphEquation}

\endmetadata

% this is the default PlanetMath preamble.  as your knowledge
% of TeX increases, you will probably want to edit this, but
% it should be fine as is for beginners.

% almost certainly you want these
\usepackage{amssymb}
\usepackage{amsmath}
\usepackage{amsfonts}

% used for TeXing text within eps files
%\usepackage{psfrag}
% need this for including graphics (\includegraphics)
%\usepackage{graphicx}
% for neatly defining theorems and propositions
 \usepackage{amsthm}
% making logically defined graphics
%%%\usepackage{xypic}

% there are many more packages, add them here as you need them

% define commands here

\theoremstyle{definition}
\newtheorem*{thmplain}{Theorem}

\begin{document}
\textbf{Theorem.}\, If\, $f(t) \equiv 0$\, for\, $t < 0$\, and\,  $\mathcal{L}\{f(t)\} := F(s)$,\, one has
$$\mathcal{L}\{f(t\!-\!t_0)\}\;=\; e^{-t_0s}F(s).$$


{\em Proof.}\, Since\, $f(t\!-\!t_0) \equiv 0$\, for\, $t < t_0$,\, the definition of Laplace transform at first gives
$$\mathcal{L}\{f(t\!-\!t_0)\}\;=\; \int_{t_0}^\infty e^{-st}f(t\!-\!t_0)\,dt.$$
The \PMlinkname{substitution}{SubstitutionForIntegration}\, $t\!-\!t_0 \,:=\, u$\, yields
$$\mathcal{L}\{f(t\!-\!t_0)\} \;=\; \int_{0}^\infty e^{-s(u+t_0)}f(u)\,du
\;=\; e^{-t_0s}\int_{0}^\infty e^{-su}f(u)\,du \;=\; e^{-t_0s}F(s).$$


\textbf{Corollary.}\, For any $f(t)$ and the Heaviside step function $H(t)$, one has
$$\mathcal{L}\{f(t\!-\!a)H(t\!-\!a)\} \;=\; e^{-as}F(s).$$

%%%%%
%%%%%
\end{document}
