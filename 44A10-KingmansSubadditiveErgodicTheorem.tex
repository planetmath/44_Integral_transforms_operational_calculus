\documentclass[12pt]{article}
\usepackage{pmmeta}
\pmcanonicalname{KingmansSubadditiveErgodicTheorem}
\pmcreated{2014-03-18 14:34:03}
\pmmodified{2014-03-18 14:34:03}
\pmowner{Filipe}{28191}
\pmmodifier{Filipe}{28191}
\pmtitle{Kingman's subadditive ergodic theorem}
\pmrecord{5}{88069}
\pmprivacy{1}
\pmauthor{Filipe}{28191}
\pmtype{Theorem}
\pmrelated{birkhoff ergodic theorem}

% this is the default PlanetMath preamble.  as your knowledge
% of TeX increases, you will probably want to edit this, but
% it should be fine as is for beginners.

% almost certainly you want these
\usepackage{amssymb}
\usepackage{amsmath}
\usepackage{amsfonts}

% need this for including graphics (\includegraphics)
\usepackage{graphicx}
% for neatly defining theorems and propositions
\usepackage{amsthm}

% making logically defined graphics
%\usepackage{xypic}
% used for TeXing text within eps files
%\usepackage{psfrag}

% there are many more packages, add them here as you need them

% define commands here

\begin{document}
Let $(M,\mathcal{A},\mu)$ be a probability space, and $f:M\rightarrow M$ be a measure preserving dynamical system. let $\phi_n:M\rightarrow \textbf{R}$, $n\geq 1$ be a subadditive sequence of measurable functions, such that $\phi_1^+$ is integrable, where $\phi_1^+=\max \{\phi,0\}$. Then, the sequence $(\frac{\phi_n}{n})_n$ converges $\mu$ almost everywhere to a function $\phi: M \rightarrow [-\infty,\infty)$ such that:
\begin{itemize}
\item[] $\phi^+$ is integrable
\item[] $\phi$ is $f$ invariant, that is, $\phi(f(x))=\phi(x)$ for $\mu$ almost all $x$, and
\item[] $$\int \phi d\mu = \lim_n \frac{1}{n} \int \phi_n d\mu= \inf_n \frac{1}{n} \int \phi_n d\mu \in [-\infty,\infty)$$
\end{itemize}
The fact that the limit equals the infimum is a consequence of the fact that the sequence $\int \phi_n d\mu$ is a subadditive sequence and Fekete's subadditive lemma. 

A superadditive version of the theorem also exists. Given a superadditive sequence $\varphi_n$, then the symmetric sequence is subadditive and we may apply the original version of the theorem.

Every additive sequence is subadditive. As a consequence, one can prove the Birkhoff ergodic theorem from Kingman's subadditive ergodic theorem.
\end{document}
