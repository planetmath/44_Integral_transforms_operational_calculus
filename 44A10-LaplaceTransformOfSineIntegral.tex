\documentclass[12pt]{article}
\usepackage{pmmeta}
\pmcanonicalname{LaplaceTransformOfSineIntegral}
\pmcreated{2014-11-07 15:36:06}
\pmmodified{2014-11-07 15:36:06}
\pmowner{pahio}{2872}
\pmmodifier{pahio}{2872}
\pmtitle{Laplace transform of sine integral}
\pmrecord{15}{40587}
\pmprivacy{1}
\pmauthor{pahio}{2872}
\pmtype{Example}
\pmcomment{trigger rebuild}
\pmclassification{msc}{44A10}
\pmsynonym{Laplace transform of sinc function}{LaplaceTransformOfSineIntegral}
\pmrelated{SubstitutionNotation}
\pmrelated{SincFunction}
\pmrelated{TableOfLaplaceTransforms}
\pmrelated{SineIntegral}

% this is the default PlanetMath preamble.  as your knowledge
% of TeX increases, you will probably want to edit this, but
% it should be fine as is for beginners.

% almost certainly you want these
\usepackage{amssymb}
\usepackage{amsmath}
\usepackage{amsfonts}

% used for TeXing text within eps files
%\usepackage{psfrag}
% need this for including graphics (\includegraphics)
%\usepackage{graphicx}
% for neatly defining theorems and propositions
%\usepackage{amsthm}
% making logically defined graphics
%%%\usepackage{xypic}

% there are many more packages, add them here as you need them

% define commands here
\newcommand{\sijoitus}[2]%
{\operatornamewithlimits{\Big/}_{\!\!\!#1}^{\,#2}}
\begin{document}
\PMlinkescapeword{integral} \PMlinkescapeword{transformation} \PMlinkescapeword{formula}
\subsection{Derivation of $\mathcal{L}\{\mathrm{Si}\,t\}$}

If one performs the change of integration variable
$$u \;=\; tx, \qquad du \;=\; t\,dx$$
in the defining \PMlinkname{integral}{DefiniteIntegral}
$$\mathrm{Si}\,t \;=\; \int_0^{\,t}\!\frac{\sin{u}}{u}\,du,$$
of the sine integral function, one obtains
$$\mathrm{Si}\,t \;=\; \int_0^1\frac{\sin{tx}}{tx}t\,dx \;=\; \int_0^1\frac{\sin{tx}}{x}\,dx,$$
getting \PMlinkescapetext{constant} \PMlinkname{limits}{UpperLimit}.\, We know (see the entry Laplace transform of sine and cosine) that
$$\mathcal{L}\left\{\frac{\sin{tx}}{x}\right\} \;=\; 
\frac{1}{s^2+x^2}.$$
This transformation formula can be integrated with respect to the parametre $x$:
$$\mathcal{L}\left\{\int_0^1\frac{\sin{tx}}{x}\,dx\right\} 
\;=\; \int_0^1\!\frac{1}{s^2+x^2}\,dx \;=\;
\frac{1}{s}\sijoitus{x=0}{\quad 1}\;\arctan\frac{x}{s} \;=\; 
\frac{1}{s}\arctan\frac{1}{s}.$$
Thus we have the transformation formula of the sinus integralis:
\begin{align}
\mathcal{L}\left\{\mathrm{Si}\,t\right\} \;=\; \frac{1}{s}\arctan\frac{1}{s}.
\end{align}

\subsection{Laplace transform of sinc function}

By the formula \;$\mathcal{L}\{f'\} = s \mathcal{L}\{f\}-\lim_{x \to 0+}f(x)$\; of the \PMlinkname{parent}{LaplaceTransform} entry,
we obtain as consequence of (1), that
$$\mathcal{L}\left\{\frac{d}{dt}\mathrm{Si}\,t\right\} \;=\; s\cdot\frac{1}{s}\arctan\frac{1}{s}-\mathrm{Si}\,0,$$
i.e.
\begin{align}
\mathcal{L}\left\{\frac{\sin{t}}{t}\right\} \;=\; \arctan\frac{1}{s}.
\end{align}


The formula (2) may be determined also directly using the definition of Laplace transform.\, Take an additional parametre $a$ to the defining integral 
$$\mathcal{L}\{\frac{\sin{t}}{t}\} \;=\; \int_0^\infty e^{-st}\,\frac{\sin t}{t}\,dt$$
by setting
$$\int_0^\infty e^{-st}\,\frac{\sin{at}}{t}\,dt \;:=\; \varphi(a).$$
Now we have the derivative \,$\varphi'(a) = \int_0^\infty e^{-st}\cos{at}\,dt$,\, where one can partially integrate twice, getting 
$$\varphi'(a) \;=\; \int_0^\infty e^{-st}\cos{at}\,dt \;=\; \frac{1}{s}-\frac{a^2}{s^2}\int_0^\infty e^{-st}
\cos{at}\,dt.$$
Thus we solve
$$\int_0^\infty e^{-st}\cos{at}\,dt \;=\; \frac{\frac{1}{s}}{1+\left(\frac{a}{s}\right)^2} \;=\; \varphi'(a),$$
and since\, $\varphi(0) = 0$, we obtain\, $\displaystyle\varphi(a) = \arctan\frac{a}{s}$.\, This yields
$$\int_0^\infty e^{-st}\,\frac{\sin t}{t}\,dt \;=\; \varphi(1) \;=\; \arctan\frac{1}{s},$$
i.e. the formula (2).

Formula (2) is derived \PMlinkname{here}{LaplaceTransformOfFracftt} in a third way.
%%%%%
%%%%%
\end{document}
